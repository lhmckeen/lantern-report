\documentclass{book}

\title{Lantern}
\newcommand{\subtitle}{Final Report for the Association of Research Libraries}
\newcommand{\booklicense}{\href{}{}}

% authors %
\author{Chris Diaz \and Lauren McKeen McDonald}
\newcommand{\authoraffiliation}{}

% Create convenient commands \booktitle and \bookauthor
\makeatletter
\newcommand{\booktitle}{\@title}
\newcommand{\bookauthor}{\@author}
\makeatother

% This utf8 declaration is not needed for versions of latex > 2018 but may
% be helpful for older software. Eventually it may not be worth keeping.
%\usepackage[utf8]{inputenc}  

\usepackage{amsmath} % Used by equations
\usepackage{hyperref}
% link options
\hypersetup{
  colorlinks=true,
  linkcolor=purple,
  urlcolor=purple,
  pdftitle={Lantern},
  pdflang={en},
  pdfcreator={LaTeX via pandoc}
}
\usepackage{booktabs}
\usepackage{longtable}
\usepackage{array}
\usepackage{graphicx}
% sets image width to text body width
\setkeys{Gin}{width=\linewidth} 
\usepackage{xcolor}
\definecolor{purple}{HTML}{663399}

% The following dimensions specify 4.75" X 7.5" content on 6 3/8" by 9 1/4"
% paper. The paper width and height can be tweaked as required and the content
% should size to fit within the margins accordingly.
%
% The (inside) bindingoffset should be larger for books with more pages. Some
% standard recommended sizes are .375in minimum up to 1in for 600+ page books.
% Sizes .75in and .875in are also recommended roughly at 150 and 400 pages.
\usepackage[bindingoffset=1in,
            left=1in, 
            right=1in,
            top=1in, 
            bottom=1in,
            paperwidth=8.27in, 
            paperheight=11.69in]{geometry}
% Here is an alternative geometry for reading on letter size paper:
% \usepackage[margin=.75in, paperwidth=8.5in, paperheight=11in]{geometry}

\usepackage[normalem]{ulem} % underlined text

\renewcommand{\contentsname}{Contents} 

% pandoc settings %
\providecommand{\tightlist}{%
  \setlength{\itemsep}{0pt}\setlength{\parskip}{0pt}}

\newenvironment{Shaded}{
    \begin{center}
    \begin{tabular}{|p{0.9\textwidth}|}
    \hline\\
    }
    { 
    \\\\\hline
    \end{tabular} 
    \end{center}
}

\newenvironment{shaded*}{
    \begin{center}
    \begin{tabular}{|p{0.9\textwidth}|}
    \hline\\
    }
    { 
    \\\\\hline
    \end{tabular} 
    \end{center}
}


% table settings

\usepackage{longtable,booktabs,array}
\usepackage{calc} % for calculating minipage widths
% Correct order of tables after \paragraph or \subparagraph
\usepackage{etoolbox}
\makeatletter
\patchcmd\longtable{\par}{\if@noskipsec\mbox{}\fi\par}{}{}
\makeatother
% Allow footnotes in longtable head/foot
\IfFileExists{footnotehyper.sty}{\usepackage{footnotehyper}}{\usepackage{footnote}}
\makesavenoteenv{longtable}

% Content Starts Here
\begin{document}
\frontmatter

% ---- Half Title Page ----
% current geometry will be restored after title page
\newgeometry{top=1.75in,bottom=.5in}
\begin{titlepage}
\begin{flushleft}

% Title
\textbf{\fontsize{48}{54}\selectfont Lantern \\}
\textbf{\large \textit{Final Report for the Association of Research
Libraries}}

% Draw a line 4pt high
\par\noindent\rule{\textwidth}{4pt}\\

% authors
\begin{flushright}

      \textbf{Chris Diaz}, \emph{Northwestern University}\\
      \textbf{Lauren McKeen McDonald}, \emph{Northwestern University}\\
  
\end{flushright}

% \vspace{\fill}
\vspace{\fill}

\end{flushleft}
  
\end{titlepage}
\restoregeometry
% ---- End of Half Title Page ----


% A title page resets the page # to 1, but the second title page
% was actually page 3. So add two to page counter.
\addtocounter{page}{2}

% The asterisk excludes chapter from the table of contents.
This report discusses the origin, development, and future of \emph{Lantern}, a
minimal computing toolkit for publishing open educational resources.

% Three-level Table of Contents
\setcounter{tocdepth}{3}
\tableofcontents

\mainmatter

\hypertarget{report}{%
\chapter{Report}\label{report}}

Northwestern University Libraries was awarded a grant from
\href{https://www.arl.org/news/arl-launches-venture-fund-to-support-research-and-development-in-member-libraries-proposals-due-february-28/}{the
Association of Research Libraries Venture Fund} to develop a digital
publishing toolkit for librarians who support open educational resources (OER)
publishing on their campuses. This report documents the outcomes and future
for our project, which we've named \emph{Lantern}, available here:
\url{https://github.com/nulib-oer/lantern}

Lantern is prototype that applies \emph{minimal computing} principles to the
production, hosting, and maintenance of OER. At its core, Lantern is a script,
template, and documentation that makes it easier to use
\href{http://pandoc.org/}{Pandoc} and \href{https://github.com/}{GitHub} to
produce and distribute open textbooks in multiple formats, such as HTML, PDF,
EPUB, and DOCX. Lantern produces OER content in formats that make it easier to
publish, preserve, and reuse by students, faculty, and librarians.

\hypertarget{why-did-we-create-lantern}{%
\section{Why did we create Lantern?}\label{why-did-we-create-lantern}}

OER programs at academic libraries typically license a publishing platform
from a vendor or develop their own with open source software. Most publishing
platforms are maintained by product managers, software developers, and other
professionals who are responsible for writing documentation, adding features,
patching security vulnerabilities, and updating software dependencies. This is
costly to sustain and creates a risk for
\href{https://en.wikipedia.org/wiki/Vendor_lock-in}{vendor lock-in}.

Lantern approaches this problem in the following ways:

\begin{itemize}
\item
  \textbf{Minimal technology stack:} Lantern was designed to use the fewest
  amount of software dependencies possible. Each dependency is cross-platform
  open source software that can run on any machine.
\item
  \textbf{Multiformat static outputs:} Lantern generates static HTML, PDF,
  EPUB, and DOCX files that can be distributed by web hosting service without
  setting up any databases, server-side application software, or content
  management systems.
\item
  \textbf{Non-proprietary:} Lantern teaches the value of non-proprietary plain
  text documents for accessibility and preservation.
\item
  \textbf{Portability:} Lantern is available on the web, in the form of a
  GitHub repository, or as a standalone production environment, where users
  can install and run the programs on their own. This gives librarians the
  assurance that they do not need to rely on any third-party vendor to produce
  multi-format OER.
\item
  \textbf{Free:} Lantern is distributed for free and under an open license.
  The first edition of Lantern is available on GitHub and assumes that the
  user has created a free GitHub account, but Lantern itself can be
  \href{https://github.com/nulib-oer/lantern/archive/refs/heads/main.zip}{downloaded
  as a .zip file} and used without a GitHub account by anyone.
\item
  \textbf{Transferable skills:} We make no promises that Lantern will be
  available and usable forever. We do, however, believe that the skills and
  knowledge introduced in Lantern's documentation provide value to librarians
  working on a variety of text-based web and digital projects.
\end{itemize}

With Lantern, we hope to demystify a few developer tools (text editors,
GitHub, and open-source software) for beginners interested in digital
publishing by connecting these technologies to concepts, such as metadata,
preservation, and information accessibility.

\hypertarget{enduring-and-barrier-free-access-to-information}{%
\section{Enduring and Barrier-free Access to
Information}\label{enduring-and-barrier-free-access-to-information}}

Lantern was inspired by \emph{minimal computing}. Minimal computing refers to
``computing done under some set of significant constraints of hardware,
software, education, network capacity, power, or other factors.'' For Lantern,
we decided to set constraints on our technology choices to minimize costs and
technical obsolescence for the preservation and dissemination of digital,
educational materials.

Lantern's emphasis on multiformat static outputs, non-proprietary file
formats, portability, and free-of-charge software are intended to support
ARL's commitment to enduring and barrier-free access to information. Markdown
is an excellent choice for this goal. We use it as the source format from
which we can generate PDF, HTML, and EPUB versions of the OER. Here is an
example of Markdown source code:

\begin{verbatim}
# Chapter Title

Introductory paragraph text with **bold** and *italic* text.

## Section Heading

Introductory paragraph for the section.

Another paragraph, but with a [link to a website](https://example.com).

### Subsection HeadingMore content, but in list form:

- list item
- list item
- list item
\end{verbatim}

\begin{quote}
The idea is to identify logical structures in your document (a title,
sections, subsections, footnotes, etc.), mark them with some unobtrusive
characters, and then ``compile'' the resulting text with a typesetting
interpreter which will format the document consistently, according to a
specified style. -\/-\/-Dennis Tenen and Grant Wythoff,
"\href{https://programminghistorian.org/en/lessons/sustainable-authorship-in-plain-text-using-pandoc-and-markdown}{Sustainable
Authorship in Plain Text using Pandoc and Markdown}"
\end{quote}

\href{https://github.com/nulib-oer/lantern/wiki}{Lantern's documentation}
walks users through the conversion and Markdown typesetting process. The basic
workflow can be completed entirely online, using only a web browser and free a
GitHub.com account. No installation of any software is required. Most users go
from manuscript to published OER in an hour.

While the basic workflow relies on GitHub to perform most of the processing
and hosting work, Lantern can be downloaded and used on any operating system,
removing GitHub as a dependency. Users who opt to run Lantern on their own
machines will have to install some software on their own. Publishing the OER
without GitHub requires access to a basic web hosting service. Since Lantern
produces static HTML, there are numerous free hosting options available, such
as
\href{https://aws.amazon.com/free/?all-free-tier.sort-by=item.additionalFields.SortRank\&all-free-tier.sort-order=asc\&awsf.Free\%20Tier\%20Types=*all\&awsf.Free\%20Tier\%20Categories=*all}{Amazon
Web Services (free tier)} and \href{https://app.netlify.com/drop}{Netlify}.

Upon completing an OER project with Lantern, users will have a website, PDF,
EPUB, and a source directory of plain text files for dissemination and
preservation of the content.

\hypertarget{librarian-review-panel}{%
\section{Librarian Review Panel}\label{librarian-review-panel}}

We wanted to make sure that Lantern, as an idea, was worth developing further.
We assembled a panel of six librarians at academic libraries around the
country to test and review an early version of Lantern. Each librarian was
paid a stipend to review the documentation, complete two tutorials, and
discuss their experiences. Some quotes from our interviews:

\begin{quote}
"You've provided an environment in which librarians who are not programmers
could experiment with using markdown. There's a lot that's unfamiliar, and I
feel like what you've done is to make it more accessible to people like me."
-- Anita Walz, Assistant Director of Open Education, Virginia Tech University
Libraries

``With Lantern, OER creation is made very simple. I could follow along and
actually get a textbook produced.'' -Kathy Clark, Director, Phillips Library,
Aurora University

``With Lantern, you can take files that someone gives you and create~a website
without having to do anything, really. I could take a 300-page Word document
textbook and easily turn it into a website." -Tim Fluhr, Head of Library
Services, Illinois Institute of Technology
\end{quote}

\hypertarget{deliverables}{%
\section{Deliverables}\label{deliverables}}

We produced the following items with ARL grant support:

\begin{itemize}
\item
  GitHub repository template for creating and publishing an OER textbook
\item
  Documentation containing instructions, guides, and recommended resources
\item
  Extensible templates for HTML, PDF, and EPUB output formats
\item
  Scripts for converting from Microsoft Word to Markdown and Markdown to
  various output formats
\end{itemize}

In addition to the above, our work involved:

\begin{itemize}
\item
  The creation of Lantern user personas
\item
  In-depth user testing and interviews with six librarians
\item
  An accessibility audit of the project website and output templates
\item
  Creation of a logo and custom book cover templates
\end{itemize}

\hypertarget{scalability}{%
\section{Scalability}\label{scalability}}

Lantern is designed to be a template for using
\href{https://pandoc.org/}{Pandoc} as a publishing tool. Pandoc is a popular
document converter that can handle dozens of file formats. While powerful,
Pandoc can be challenging to learn without a strong familiarity with command
line programs and plain text formats. Lantern provides the fundamentals
(instructions, explanations, and templates) from which users can explore
numerous customizations for their projects. Our hope is that Lantern can be
used as-is for open textbook projects, but also be modified for a variety of
text-based digital web projects.

We have ideas on how Lantern could be used by content creators beyond the open
textbook paradigm. While the primary audience is currently librarians and the
primary output is OER, Lantern's methods could be used to easily turn meeting
notes, slide decks, conference proceedings, presentations, personal writing
projects and more into static websites and PDFs. In other words, we can see
how the variety of use cases could expand Lantern's use beyond its original
intention.

\hypertarget{sustainability}{%
\section{Sustainability}\label{sustainability}}

We have just started using Lantern on OER projects at Northwestern University,
so it is too soon to tell if Lantern as a software is sustainable. We,
personally, are committed to using and maintaining Lantern on Northwestern OER
projects for as long as Lantern meets the needs of our OER authors.

We are confident, however, that the publication outputs produced by Lantern
can be sustained in terms of access and preservation by libraries. Lantern
outputs produce multiple formats (Markdown, HTML, PDF, EPUB, and DOCX) as
static files. The flexibility of multiple, non-proprietary output formats
gives librarians some assurances future software will be able to read and
compile the files for use.

\hypertarget{future}{%
\section{Future}\label{future}}

We have three priorities for Lantern: (1) use Lantern to publish OER textbooks
by Northwestern University authors, (2) extend Lantern to support more use
cases, and (3) promote Lantern at academic librarian conferences.

First, we are currently using Lantern on OER projects Northwestern University
Libraries supports, developing and releasing new features as needed. We began
by migrating two OER projects from \href{https://bookdown.org/}{Bookdown} to
Lantern\footnote{Bookdown is a major influence for the development of Lantern.
  We strongly recommend Bookdown to any users who are familiar with the R
  programming language. We developed Lantern as an alternative to Bookdown in
  order to provide a similar option to user who are unfamiliar with the R
  programming language (and therefore do not need the features Bookdown
  provides to R users).}:
\href{https://emps.northwestern.pub/}{\textbf{\emph{Empirical Methods in
Political Science: An Introduction}}} by Jean Clipperton and
\textbf{\emph{Introduction to Material Science and Engineering}} by Kenneth
Shull, Jonathan Emery, and Jacob Kelter.

Second, we were thrilled to discover new use cases for Lantern from the
Librarian Review Panel interviews, including:

\begin{itemize}
\item
  \emph{Convert LaTeX to EPUB:} LaTeX is a popular typesetting language for
  STEM disciplines, but PDFs generated from LaTeX are not accessible. EPUB,
  however, is a more accessible format than LaTeX and PDF. Lantern could be
  extended to support a workflow that generates EPUB from LaTeX source.
\item
  \emph{Create a public website from a Google Doc:} Google Docs are powerful
  tools for collaboratively generating a document from a learning activity.
  Instructors might want to generate a more stable and sharable version of the
  Google Doc by converting it into a static website. Google Docs can export to
  .DOCX or .ODT formats, which Lantern can convert to Markdown, then HTML.
\end{itemize}

Third, we will be promoting Lantern at several academic librarian conferences
in the spring and summer of 2022. We hope to connect with potential users on
GitHub. Lantern users who have questions can use the
\href{https://github.com/nulib-oer/lantern/discussions}{discussion forum} or
\href{https://github.com/nulib-oer/lantern/issues}{issue tracker} to report a
bug or request a feature.

We believe that Lantern is aligned with the movement working toward
\href{https://investinopen.org/about/}{an open infrastructure for scholarly
communication} and has the potential to be an important option for
\href{https://mindthegap.pubpub.org/}{open source publishing tools}. We hope
Lantern can provide an on-ramp to more librarians interested in applying
minimal computing techniques to open education initiatives.

\hypertarget{appendix}{%
\chapter{Appendix}\label{appendix}}

\hypertarget{summary-of-grant-expenses}{%
\section{Summary of Grant Expenses}\label{summary-of-grant-expenses}}

In our original proposal we budgeted \$2,007 for an in-person training event.
With the switch to remote work, we altered our budget to commit \$2,500 to pay
honoraria of \$500 each to librarians from across the country to do an
in-depth review of the toolkit.

The review was followed by a recorded Zoom interview in which received
feedback, comments, and suggestions. In the end, we recruited six librarians
for the review, increasing this line item to \$3000.

Fortunately, we hired a Northwestern Libraries colleague to complete the
accessibility audit for \$1000, which was much less expensive than hiring a
specialized company.

In the end, \$441.26 was left unspent due to unforeseen student employee
scheduling issues.

\begin{longtable}[]{@{}
  >{\raggedright\arraybackslash}p{(\columnwidth - 4\tabcolsep) * \real{0.2738}}
  >{\raggedright\arraybackslash}p{(\columnwidth - 4\tabcolsep) * \real{0.5119}}
  >{\raggedright\arraybackslash}p{(\columnwidth - 4\tabcolsep) * \real{0.1905}}@{}}
\toprule
\begin{minipage}[b]{\linewidth}\raggedright
Item
\end{minipage} & \begin{minipage}[b]{\linewidth}\raggedright
Description
\end{minipage} & \begin{minipage}[b]{\linewidth}\raggedright
Cost
\end{minipage} \\
\midrule
\endhead
Research and Development Assistant & Research and test software for possible
inclusion in the toolkit & \$1,639.62 \\
Instructional Design and Communication Assistant & Assist in the creation of
user-facing documentation, tutorials, and guides & \$3,279.25 \\
Document Production Assistant & Assist with PDF, HTML, EPUB, and DOCX
templating & \$1,639.63 \\
Employee Fringe Benefits & Standard benefits paid to graduate student
employees & \$67.00 \\
Accessibility Audit & Specialist to provide an accessibility audit on sample
output files & \$1,000.00 \\
Graphic Design services & To develop a logo for the completed project website
& \$500.00 \\
Librarian Review Panel & Six librarians selected to review and provide
feedback on an pre-release version of the tookit & \$3,000.00 \\
TOTAL SPENT & & (\$11,125.50) \\
TOTAL AWARDED & & \$11,566.76 \\
REMAINDER & & 441.26 \\
\bottomrule
\end{longtable}

\backmatter
\end{document}
